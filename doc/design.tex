\documentclass[10pt]{article}
\usepackage[utf8]{inputenc}
\usepackage{geometry}
\usepackage{wrapfig}
\geometry{legalpaper, portrait, margin=1in}

\title{Basic to Verilog Compiler }
\author{Richard P Martin}
\date{March 2019}

\usepackage{natbib}
\usepackage{graphicx}

\begin{document}

\maketitle

\section*{Abstract}


The Basic to Verilog compiler, {\em basic2verilog} is an introduction
to High Level Synthesis (HLS)  using the BASIC language. BASIC is a simple
language, yet quite powerful in its ability to create meaningful
programs by programmers with limited experience and modest computing
infrastructure.

The goal of the basic2verilog is to allow a BASIC program to run in a
Verilog simulator using the {\em synthesizable} Verilog. It is 
Verilog sub-set, and style of design, which can be efficiently realized on
an FPGA. Using BASIC allows illustration of simple techniques without
getting bogged down in more advanced language constructs such as
inheritance and reflection.

Restricting the target to synthesizable Verilog must numerous contrainsts
on the architecture of the mapping but exposes the challenegs of
FPGAs as a target. 
For example, Verilog loops in synthesis create copies on the
FPGA surface, they do not create a structure that results in
repeated execution. Thus, other Verilog constructs must to used
to realize looping as a repeated execution with uknown compile-time bounds. 

Basic2verilog is built using the PLY (Python Lex Yacc) compiler framework.~\cite{ply}. This was done because PLY already had an BASIC interpreter and
parser as examples. 

\section{Project Organization}

\begin{verbatim} 

  src/                   // source code for the Verilog Generator
      basveri.py         // top level program which can either run the 
                         // BASIC interpreter or generate Verilog
      basic2verilog.py   // main python code that has an interpreter 
                         // and Verilog generator

      baspars.py         // Parser 
      basiclex.py        // Lexical Analyzer                  

  examples/              // Example Basic programs to illustrate PL techniques 
      ifthen.bas         // Example of IF THEN ELSE
      gosub.bas          // Example of GOSUB 
      gosub2.bas         // Nested GOSUB
      array.bas          // Uni-Dimensional Array
      dim.bas            // 2-Dimensionsal Arrays 
      loop2.bas          // Nested Loops 

  benchmarks/            // Benchmark programs to measure performance
    
  doc/                   // Documentation 
      design.tex         // Theory and Design documentation

\end{verbatim}

\section{Theory Of Design}

A BASIC program is parsed into a Abstract Syntax Tree (AST). The AST
is built from lists-of-lists. The first 

\subsection{Control and DataFlow}

Control flow in the Verilog is represented as bit-vector (BV) rather than
a program counter (PC). That is, each basic block is controled by a
one-hot encoded bit. If the bit is '1', the block is executed.
Using a BV needs more memory than a PC, but the upside is that
it makes parallel realization much more straightforward. It also maps
to FPGA block in a more natural format as each LUT can hold logic
and a bit. 

Writes to each variable are controlled by an {\tt always} statement
per variable as well as a register or array.
That is, each BASIC variable  maps to a combination of an
{\tt always} block and the state maintaing the variable. The
{\tt always} uses the control bits from the BV to arbitrate writes.
That is, depending on which control bits are active, different

Because synthesizable Verilog allows many readers on each clock tick,
we reads do not require arbitration. 

All writes use the 

\section{Data Structures and Translation}


This section describes the main data structures in the compiler:
The parse tree, the statement graph, the control-flow graph, and
the list of variables.

\subsection{List of Variables}

This is just a Python map of the list off all the variables with some
simple typing information for the dimensions of arrays. Only 1D and 2D
arrays are supported as these map directy to Verilog

\subsection{Parse Tree}

The parse tree holds a parsed representation of the BASIC code as a list
of lists. Each statement is a node in the tree. Edges represent textual
flow of the structure of the program, rather than control flow. 

\subsection{Statement Graph} 

The first pass over the parse tree produces a statement graph. Each node is of the statement type. Control flow is represented as edges in the statement graph
as control flow between statements. There are different edge types for
different types of control flow, for example the target of a {\tt gosub}
or a {\tt for} statement. 

\subsection{Control Flow Graph}

The control flow graph (CFG) is a lower-level representation of the BASIC
program than the statement graph. Each node represents a control point,
such as the test of a loop condition, or a bubble needed post the store
of a variable. 


\bibliographystyle{plain}
\bibliography{design.bib}
\end{document}


