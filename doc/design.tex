\documentclass[10pt]{article}
\usepackage[utf8]{inputenc}
\usepackage{geometry}
\usepackage{wrapfig}
\geometry{legalpaper, portrait, margin=1in}

\title{Basic to Verilog Compiler }
\author{Richard P Martin}
\date{March 2019}

\usepackage{natbib}
\usepackage{graphicx}

\begin{document}

\maketitle

\section*{Abstract}

This document describes the theory of the Basic to Verilog, {\em basic2verilog},compiler. The goal of the compiler to translate a program in BASIC to run on a a Verilog simulator. Using BASIC allows illustration of simple techniques to translate high level language (HLL) constructs into Verilog without getting bogged down in more advanced language constructs such as inheritance and reflection.

In addition, the generated Verilog code should synthesize on an FPGA. This synthesis constraint means the Verilog code must follow the design style needed for code synthesis. 

Basic2verilog is built using the PLY (Python Lex Yacc) compiler framework.~\cite{ply}.

\section{Project Organization}

\begin{verbatim} 

  src/                   // source code for the Verilog Generator
      basveri.py         // top level program which can either run the 
                         // BASIC interpreter or generate Verilog
      basic2verilog.py   // main python code that has an interpreter 
                         // and Verilog generator

      baspars.py         // Parser 
      basiclex.py        // Lexical Analyzer                  

  examples/              // Example Basic programs to illustrate PL techniques 
      ifthen.bas         // Example of IF THEN ELSE
      gosub.bas          // Example of GOSUB 
      gosub2.bas         // Nested GOSUB
      array.bas          // Uni-Dimensional Arrays 
      dim.bas            // 2-Dimensionsal Arrays 
      loop2.bas          // Nested Loops 

  benchmarks/            // Benchmark programs to measure performance
    
  doc/                   // Documentation 
      design.tex         // Theory and Design documentation

\end{verbatim}


\section{Theory Of Design}

\subsection{Control and DataFlow}

\section{Data Structures and Translation}

\subsection{Parse Tree}

\subsection{Statement Graph} 

\subsection{Control Flow Graph}

\bibliographystyle{plain}
\bibliography{design.bib}
\end{document}


